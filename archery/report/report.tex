\documentclass[8pt]{article}
\usepackage[margin=0.75in]{geometry}
\usepackage{fancyhdr}

\usepackage{longtable}

\pagestyle{fancy}
\fancyhf{}
\lhead{CS324 - Computer Graphics}
\rhead{Thomas Archbold, 1602581}

\title{CS324 Computer Graphics Coursework 2018}

\begin{document}

\maketitle

\begin{center}
    Thomas Archbold \\
    1602581 \\
    Unversity of Warwick
\end{center}

\section*{Features of the solution}
% First person shooter
% Animation of bow and arrow in realistic manner
% Different game modes - timed, high score, limited arrows
% Alter number of targets
This solution features an archery game in the style of a first-person shooter.
The player is equipped with a simple bow and a limited number of arrows, which
can be used to shoot at the targets floating in the field before them. The main
objectives available to pursue in the game are to obtain a high score, by
hitting targets as close to their centre as possible, and to hit all the targets
(anywhere) as quickly as possible. There are several aspects to the gameplay
that the player is able to change while playing, and these include the number of
arrows in the player's quiver, the number and position of targets, and the
difficulty level. In particular, difficulty ranges from having stationary
targets, to targets which move horizontally only, to targets which move both
horizontally and vertically.

\section*{Main design aspects}

% simulation
% collision detection
\section*{Specific OpenGL/GLSL features are employed}
The solution mainly uses the fixed function pipeline to draw the various shapes
to the screen, although using the code provided as part of the labs, the option
is open for shaders to be written and used. 

\section*{Compiling and running}
The makefile provided has been modified slightly from the labs in order to make
compilation as simple as possible. All that is needed is to enter the
\texttt{make} command at the prompt and the entire solution will be built. This
uses make's pattern matching to iterate through all \texttt{.cpp} files in the
directory and compile them into object files, to produce the final executable
\texttt{archery}. It is then run in the normal way from the command prompt.
Since there is support for GLSL shaders, different ones may be selected by
providing them as command line arguments. So to use the toon shaders from the
labs, issue the command \texttt{./archery toon.vert toon.frag}.

\section*{Using the application}
Once the application has started the user is free to start firing arrows.
Various important pieces of information are displayed in the heads up display,
such arrows remaining and the current score. The user may also press the 'h' key
for help, which displays the controls and brief instructions on how to play the
game. The player is free to play however they want; they may go for one
objective, both, or neither, and in addition switch between the different modes
using the keys found in the help messages.

\section*{Further ideas}
\section*{Known bugs/limitations}

\end{document}
